\begin{abstract}
\noindent Lo scopo della tesi è studiare gli effetti delle fluttuazioni delle strutture proteiche sul grado di entanglement delle stesse. Si sono inizialmente presentate le caratteristiche generali delle proteine sottolineando aspetti della loro struttura e composizione di particolare interesse per questo lavoro di tesi. Successivamente si è introdotto l'entanglement topologico di strutture proteiche con particolare attenzione al calcolo dell'entanglement gaussiano, parametro capace di valutarne matematicamente l'entità. Si sono poi trattati i principali metodi di analisi delle strutture proteiche ed i modelli a rete elastica analizzando nel dettaglio il modello anisotropo in seguito utilizzato. Infine si sono presentate le modalità e i risultati dell'analisi svolta su due specifiche proteine.
\end{abstract}